\chapter{Discussion}

\section{Hardware Precision}

The current FPGA-based control system for quantum laser control demonstrates proof-of-concept functionality and lays a foundation for future applications. However, test results reveal consistent discrepancies between theoretical and measured values. These errors stem primarily from the rounding mechanism. \autoref{fig:block_diagram} indicates that data and address inputs in the waveform table memory are floored to the nearest integer, which discards important fractional components and limits resolution.

Waveforms are generated from predefined tables that are externally computed and then loaded into the system. This method simplifies on-chip logic and resource management by avoiding complex dynamic computations. However, using a fixed 16-bit integer representation introduces quantization errors when multiple write operations produce a complete waveform. The resulting trade-off between simplicity and precision underscores the limitations of the current design. To improve accuracy, enhanced rounding methods—such as midpoint rounding or adaptive precision schemes—could reduce truncation errors. A shift toward dynamic waveform generation, where only key parameters (amplitude, frequency, phase) are provided, may further minimize data transfer overhead and yield waveforms that better match their theoretical definitions.

\section{Hardware Error Handling}

The system utilizes a basic error detection framework that effectively identifies invalid parameters. When an error occurs, the system sets a status flag for the controlling processor to read. However, it relies on a passive notification system rather than active error recovery. This approach depends entirely on a dedicated software interface to resolve issues, which can lead to complications if the software is not properly implemented or if parameter conversions fail.

For instance, the pulse channel's definition memory assumes that waveform parameters are provided in a consecutive, continuously incrementing order. There is no built-in mechanism to verify or correct such an order. The module directly maps the provided address to the memory. This design decision exposes the system to potential data inconsistencies when parameters are not correctly sequenced. Future development should introduce a mechanism to enforce cumulative writes of pulse parameters. 

Additionally, although allowing users to write to any memory location increases flexibility, it also raises the risk of overlapping writes. If memory address management is not tightly controlled, the chance of overriding existing data becomes significant. Addressing this vulnerability is pivotal to safeguard the system's stability and reliability.

% Subjected for removal if alternate firmware implemented
\section{Interface Latency}
% TODO: this is only the idea for new implementation... will get modified a lot after the new interface is done
Many user-level interfaces are implemented in Python on a separate PC, requiring frequent serial writes to the hardware system. Each UART serial write transmits only 8 bits of data from the host to the FPGA. This limited throughput results in extended read and write times. Tests have shown that reading the entire waveform memory takes three to four seconds on average. Such delays become a significant bottleneck, especially when multiple memory accesses occur during operation.

To mitigate this issue, sophisticated C firmware could replace much of Python's functionality. This firmware would reside on Xilinx's processing system, fully leveraging Zynq's high-speed AXI interface to accelerate data transfers. Instead of relying on Python functions to relay user parameters, the processing system would accept input directly from a command-line interface. Users could manually type parameters or design scripts to automate their entry.