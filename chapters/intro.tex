\chapter{Introduction}

Trapped-ion quantum computing stands at the forefront of modern quantum technology, promising unparalleled computational capabilities through the manipulation of individual atomic ions. A cornerstone of these systems is the laser control unit, which must deliver precisely tailored optical pulses with extreme timing accuracy \cite{manychanfpgactrlsys}. Motivated by recent advances in both quantum experiments and control hardware, this thesis presents a novel laser control system specifically designed for trapped-ion quantum computing.

This hardware design, implemented on a Xilinx Zynq FPGA platform, harnesses the combined power of high-speed and low-speed digital-to-analog converters (DACs) to achieve synchronous control across 32 discrete laser channels. With a frequency of 100 MHz, the system meets the rigorous temporal precision required for high-fidelity qubit manipulation. Such demanding performance is essential, as even slight deviations in timing can significantly impact the accuracy and repeatability of quantum operations \cite{manychanfpgactrlsys}.

The design approach is both well-structured and modular, ensuring a robust foundation for development. Initially, a single pulse channel will be developed and rigorously evaluated as a proof-of-concept. This initial validation is crucial as it demonstrates the feasibility and effectiveness of the design. Once the single channel is successfully validated, it will be scaled into a comprehensive system by integrating custom programmable logic blocks. These blocks are designed to efficiently deliver stable DC voltage control and generate dynamic, high-frequency pulsed waveforms. This capability ensures versatile and precise signal manipulation. The scalability and modularity of the design not only meet current experimental demands but also provide a robust framework that can adapt to future challenges in quantum computing \cite{programmablesoc4ctrlexperiment}. This adaptability is particularly important as it allows the system to evolve with advancements in the field, ensuring long-term relevance and utility.

Central to the system's performance is its effective exploitation of the FPGA's built-in features. By incorporating custom memory IP, high-speed internal communication interfaces, and an embedded ARM CPU core, the design achieves remarkable throughput and robustness. This careful coordination of resources ensures precise hardware timing and optimal utilization. This strategic use of FPGA capabilities underscores the system's ability to deliver high performance and reliability.